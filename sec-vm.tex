\section{使用 Linux 虚拟机}%
\label{sec:linux}

\subsection{获取虚拟机}%
\label{ssub:vm-download}

此虚拟机的安装和配置参见 \S\ref{sub:vbox-vm-u18-install-set}.

\subsection{内存分配}%
\label{ssub:vbox-set-memory}

在虚拟机的系统设置中,可以设置虚拟机所用内存大小。设置内存请遵循以下原则:
\begin{enumerate}
    \item 虚拟机的内存不能超过主机物理内存的一半;
    \item 如要顺畅地运行,最好设置不小于 2048 MB 的虚拟机的内存。
\end{enumerate}

\subsection{硬盘设置}%
\label{ssub:vbox-set-vdi}

如果虚拟机运行在固态硬盘上,在虚拟机的存储设置中的属性处,选择固态驱动器。

\subsection{设定共享目录}%
\label{ssub:vbox-set-share-folder}

当虚拟机需要与主机系统交换文件时,就可以设定共享目录。
% 重启虚拟主机或者运行命令 \lstinline{sudo systemctl daemon-reload},
% 共享目录生效。
共享目录的设定详情参见附录 \S\ref{ssub:vbox-set-share-folder-a}.
% \begin{tip}\label{tip:vbox-share-folder-activate}
%     如果曾经运行过 \S\ref{ssub:vbox-ganx-vboxsf} 中的命令授予用户权限,
%     可以运行命令 \lstinline{sudo systemctl daemon-reload},
%     共享目录在未重启虚拟机的情况下马上生效。
% \end{tip}

\subsection{调整虚拟机窗口大小}%
\label{ssub:vbox-set-window-size}

此虚拟机的窗口大小缺省设置为 \(1360 \times 768\).
如果需要更大的窗口,可以在打开虚拟机后,于窗口顶部菜单选择
\menu{视图 > 自动调整尺寸},然后最大化虚拟机窗口。
(在 Linux 和  Windows 上通常为点击右上角的方框,在 Mac 上点击左上的放大键)
\begin{figure}[!htbp]
  \centering
  \includegraphics[width=1.0\textwidth]%
  {media/shots/vm/vlcsnap-2020-01-28-01h24m55s147-vm.png}
  \caption{\menu{视图 > 自动调整尺寸}}%
  \label{fig:vm-display}
\end{figure}

调整虚拟机窗口还有其它方法,我们这里提供两种:
\begin{enumerate}
    \item 按组合键 Host + F.(组合键在 Linux 和 Windows 上通常为右边的
        Ctrl, 在 Mac 上通常为左边的 Command 键)
    \item 将鼠标移动到窗口边缘,调节窗口大小。
\end{enumerate}

\subsection{Virtualbox 的其它设置}%
\label{ssub:vbox-others}
\begin{enumerate}
    \item 设置虚拟机的共享剪贴板选项为 \emph{双向};
    \item 在 \textbf{设置 --> 存储}
        菜单中,设置虚拟硬盘是放置在固态硬盘还是机械硬盘上。
    \item 设置用户界面,隐藏菜单以增加虚拟机窗口大小。
\end{enumerate}
\newpage
