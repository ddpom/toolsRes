%!TEX program = xelatex
%!TEX encoding = UTF-8 Unicode

\documentclass{ctexart}

% 待完成:将“文档标题” 改为 "LaTeX 基础练习"。
% 注意:不要漏输了 LaTeX 之后的空格。
\title{标题}

% 待完成:输入自己的名字,替换"作者名字"。
\author{作者名字}

\date{\today}

\begin{document}

\maketitle

\section{排版文字}\label{sec:text}

\subsection{中文输入}\label{sub:basic-text-chinese}

% 待完成学习:在“普通字体:” 后面输入 "你好,世界!". 
普通字体:

% 待完成学习:在“加粗字体:” 后面输入 "\textbf{你好,世界!}". 
加粗字体:

% 待完成学习:在“意大利斜体:” 后面输入 "\textit{你好,世界!}". 
意大利斜体:

\subsection{英文输入}\label{sub:basic-text-english}

% 待完成学习:在“普通字体:” 后面输入 "Hello, world!". 
普通字体:

% 待完成学习:在“加粗字体:” 后面输入 "\textbf{Hello, world!}". 
加粗字体:

% 待完成学习:在“意大利斜体:” 后面输入 "\textit{Hello, world!}". 
意大利斜体:

\section{排版数学式子}\label{sec:math}

\subsection{行内和行间公式}\label{sub:math-eq}

% 待完成学习:在"行内数学式子:" 后面输入 "$ 1 + 2 = 3 $".
行内数学式子方法 1:

% 待完成学习:在"行内数学式子:" 后面输入 "\(1 + 2 = 3\)".
行内数学式子方法 2:

% 待完成学习:在"行间数学式子:" 后面输入 "\[ 1 + 2 = 3 \]".
行间数学式子方法 1:

% 不推荐使用方法 2. 待完成练习:应用前面的知识,对``不推荐使用''加粗以警示。
行间数学式子方法 2 (不推荐使用):$$ 1 + 2 = 3 $$

\end{document}
