%!TEX TS-program = xelatex
%!TEX encoding = UTF-8 Unicode

\documentclass[cn,11pt, simple]{elegantbook}

\title{研究工具简单教程: Linux、Vim、\LaTeX{}、Git}
\subtitle{基于作者配置好的 Virtualbox 虚拟机}

\author{甘湘华}
\institute{西南财经大学}
\date{\today}
\version{0.01}

\extrainfo{本教程基于已经配置好所用软件的虚拟机,力求尽快让技术小白能够用
    {\rm \LaTeX{}} 写研究论文,以及用~{\rm \lstinline{Git}}
    工具来进行版本控制。
    需要声明的是,作者本人也是小白一枚,请读者批判地学习本教程。
    不当以及需改进之处,请不吝指出,谢谢!}

\logo{logo.png}
\cover{cover.jpg}

\begin{document}

\maketitle

\tableofcontents
% \thispagestyle{empty}

\mainmatter
\hypersetup{pageanchor=true}

\chapter{Virtualbox 虚拟机使用教程}
\label{cha:virtualbox-tips}

\section{参考资料}%
\label{sec:virtualbox-refs}

Virtualbox 虚拟机软件参考资料网址:
\begin{itemize}
    \item 官方网址: \href{https://www.virtualbox.org/}{https://www.virtualbox.org/}
    \item 官方论坛: \href{https://forums.virtualbox.org/}{https://forums.virtualbox.org/}
    \item 网络上的非官方网址和非官方教程:
        \begin{itemize}
            \item 如果官方网址下载虚拟机软件慢的话,
                可以在
                \href{https://mirror.tuna.tsinghua.edu.cn/virtualbox/6.0.14/}
                {清华镜像}下载。
            \item \href{https://www.jianshu.com/p/bfb4f4415411}{安装
                virtualbox 教程}\footnote{本教程使用的虚拟机版本是 6.0.14}
                \footnote{本教程作者已经感谢该文作者。}
        \end{itemize}
\end{itemize}

\section{使用教程及技巧}%
\label{sec:virtualbox-tips}

\subsection{设定共享目录}%
\label{sub:virtualbox-tip-share-folder}
\begin{figure}[!htbp]
  \centering
  \includegraphics[width=0.9\textwidth]{screenshots/Screenshot from
  2019-11-29 17-52-45.png}
  \caption{设定共享目录 --- 示意图 1}
\end{figure}
\begin{figure}[!htbp]
  \centering
  \includegraphics[width=0.9\textwidth]{screenshots/Screenshot from
  2019-11-29 17-53-28.png}
  \caption{设定共享目录 --- 示意图 2}
\end{figure}
\begin{figure}[!htbp]
  \centering
  \includegraphics[width=0.9\textwidth]{screenshots/Screenshot from
  2019-11-29 17-54-41.png}
  \caption{设定共享目录 --- 示意图 3}
\end{figure}
\begin{figure}[!htbp]
  \centering
  \includegraphics[width=0.9\textwidth]{screenshots/Screenshot from
  2019-11-29 17-55-59.png}
  \caption{设定共享目录 --- 示意图 4}
\end{figure}
\begin{figure}[!htbp]
  \centering
  \includegraphics[width=0.9\textwidth]{screenshots/Screenshot from
  2019-11-29 17-59-22.png}
  \caption{设定共享目录 --- 示意图 5}
\end{figure}

\newpage
\subsection{调整虚拟机窗口大小}%
\label{sec:virtualbox-tip-window-size}

至少有三种方法:
\begin{enumerate}
    \item 最大化虚拟机窗口。(在 Linux 和 Windows 上通常为点击右上角的方框,
        在 Mac 上点击左上的放大键)
    \item 按组合键 Host + H.(组合键在 Linux 和 Windows 上通常为右边的
        Ctrl, 在 Mac 上通常为左边的 Command 键)
    \item 用鼠标拖动窗口,调节窗口大小。
\end{enumerate}

\chapter{Linux 使用教程}%
\label{cha:linux-tips}

\section{Linux 的简单介绍}%
\label{sec:linux-intro}

\subsection{Linux 的系统配置}%
\label{sub:linux-settings}

在 Unix/Linux/Mac 系统上,软件有它们各自的配置文件,通常以 ~.  开头,
管它们叫 \emph{点文件}~(~ \texttt{dotfiles}).
这些点文件数量多,并且所在的位置还有所不同,管理起来并不是一个容易的事情。
一个方法是把所有的这些配置文件放进一个文件夹,对每个配置文件用命令
\lstinline{ln -s} 链接到原来的位置。
这个文件夹可以用 git 仓库进行管理,
可以方便我们把一台电脑上的配置,同步到另外的电脑上。

作者配置的虚拟机上使用
\href{https://github.com/anishathalye/dotbot}{dotbot} 技术来管理 dotfiles.
在本虚拟机的 \lstinline{~/.df/} 目录中有两个文件夹: \lstinline{dotfiles}
中的配置文件是所有电脑中都共同拥有的; \lstinline{dotfiles-local}
中的配置文件是某个电脑所特有的。
建议本教程的读者在未掌握 \lstinline{dotfiles}
的配置前,不要更改这两个文件夹中的任何文件。

\subsection{本教程所用虚拟机的配置}%
\label{sub:virtualbox-guest-settings}

待续

\subsection{使用本虚拟机的注意事项}%
\label{sub:virtualbox-guest-ganx}

\subsubsection{同步配置文件的 git 仓库}%
\label{ssub:virtualbox-guest-ganx-conf}

本虚拟机的配置文件主要由以下两个仓库控制:
\begin{itemize}
    \item \href{https://github.com/ddpom/dotfiles}
        {https://github.com/ddpom/dotfiles}
    \item \href{https://github.com/ddpom/dotfiles-local}
        {https://github.com/ddpom/dotfiles-local}
\end{itemize}

\subsubsection{设置配置文件仓库的更新提示}%
\label{ssub:virtualbox-guest-ganx-conf-update-notification}

两个仓库不时更新的,更新配置文件的提交实现后,会发出更新提示。
如果你 \lstinline{watch (关注)} 了这两个仓库,就可以收到更新提示。
设置步骤如下:
\begin{enumerate}
    \item 访问 \href{https://github.com/ddpom/dotfiles}
        {https://github.com/ddpom/dotfiles}, 点击右上方
        \lstinline{watch (关注)} 旁边的下拉菜单,选择 \lstinline{watching}.
        如果你觉得这个仓库还有用的话,顺便点一下右边的 \lstinline{star}.
    \item 访问 \href{https://github.com/ddpom/dotfiles-local}
        {https://github.com/ddpom/dotfiles-local},
        点击右上方 \lstinline{watch (关注)} 旁边的下拉菜单,
        选择 \lstinline{watching}.
        如果你觉得这个仓库还有用的话,顺便点一下右边的 \lstinline{star}.
    \item 访问 \href{https://github.com/settings/notifications}
        {https://github.com/settings/notifications},
        配置你接收提示的选项。缺省设置下,你会在电子邮件里收到你关注的仓库的
        提交。你还可以打钩 \lstinline{Web and Mobile} 选项。
\end{enumerate}

\subsubsection{更新配置文件 --- 手动方式}%
\label{sec:update-dotfiles}

\begin{itemize}
    \item 运行下面几个命令,更新 \href{https://github.com/ddpom/dotfiles}
        {https://github.com/ddpom/dotfiles}:
        \begin{lstlisting}[escapeinside=``]
    cd ~/.df/dotfiles
    git checkout .
    git pull --rebase
    git submodule update --remote --recursive
    ./install
        \end{lstlisting}
    \item 运行下面几个命令,更新
        \href{https://github.com/ddpom/dotfiles-local}
        {https://github.com/ddpom/dotfiles-local}:
        \begin{lstlisting}[escapeinside=``]
    cd ~/.df/dotfiles-local
    git checkout .
    git pull --rebase
    git submodule update --remote --recursive
    ./install
        \end{lstlisting}
\end{itemize}

%\section{Linux 的学习方式}%
%\label{sec:linux-learning}
%
%待续
%
\section{Linux 的常用命令}%
\label{sec:linux-commands}

常用命令:
\begin{enumerate}
    \item \lstinline{cd}: 进 入某个目录; 例子:
\begin{lstlisting}[ escapeinside=``]
    cd ~/.ssh/
\end{lstlisting}
    \item \lstinline{mv}: 1. 修改文件或这目录的名字; 2.
        移动文件或者整个目录; 例子:
\begin{lstlisting}[escapeinside=``]
    cd ~/.ssh/
    mv github_yourname_ganx github_ddpom_ganx
    mv github_yourname_ganx.pub github_ddpom_ganx.pub
\end{lstlisting}
\end{enumerate}

\chapter{Vim 使用教程及技巧}%
\label{cha:vim-tips}

\section{参考资料}%
\label{sec:vim-refs}

\begin{itemize}
    \item 官方网址: \href{https://www.vim.org/}{https://www.vim.org/}
    \item 问答网站:
        \href{https://vi.stackexchange.com/}{https://vi.stackexchange.com/}
    \item vim 教程:
        \begin{itemize}
            \item \lstinline{vimtutor}
                \footnote{在终端中,输入 \lstinline{vimtutor} 命令学习 Vim
                的基本用法。}
            \item \href{https://github.com/wsdjeg/vim-galore-zh_cn}
                {Vim 从入门到精通}
            \item \href{https://item.jd.com/12056490.html}
                {Vim 使用技巧 (Pratical Vim) 第二版}
        \end{itemize}
\end{itemize}

\section{Vim 的配置}%
\label{sec:vim-intro}

\subsection{Vim 的配置文件}%
\label{sub:vim-config}

本虚拟机中的配置文件为:
\begin{lstlisting}[escapeinside=``]
    ~/.df/dotfiles/vimrc
    ~/.df/dotfiles-local/gvimrc
    ~/.vimrc_customized
    ~/.gvimrc_customized
\end{lstlisting}
\lstinline{vimrc} 和 \lstinline{gvimrc} 由
\ref{ssub:virtualbox-guest-ganx-conf} 中的两个 git 仓库控制,请勿修改。
你对本台虚拟机的特殊设置可以在 \lstinline{.vimrc_customized} 和
\lstinline{.gvimrc_customized} 中配置。

%待续
%%Vim 的学习曲线(To do)。
%
%\section{Vim 的学习方式}%
%\label{sec:vim-learning}

\chapter{\LaTeX{} 使用教程}%
\label{cha:latex-tips}

\section{参考资料}%
\label{sec:latex-refs}

\LaTeX{} 参考资料网址:
\begin{itemize}
    \item 官方网址: \href{https://www.ctan.org/}{https://www.ctan.org/}
    \item 问答网站:
        \href{https://tex.stackexchange.com/}{https://tex.stackexchange.com/}
    \item 软件下载:如果官方网址下载 \lstinline{texlive.iso} 慢的话,可以在
    \href{https://mirrors.tuna.tsinghua.edu.cn/CTAN/systems/texlive/Images/}
                {清华镜像}~下载。
    \item 数学符号: \href{books/latex-math-symbols.pdf}{速查表}~ ~
        \href{http://mirrors .ustc.edu.cn/CTAN/info/symbols/comprehensive/symbols-a4.pdf}
        {详细列表}
        \footnote{如果 \lstinline{git clone} 了本仓库,
        可以在本仓库的 \lstinline{books}
        目录中阅读此电子书:
        \href{books/symbols-a4.pdf}{数学符号详细列表} 。
        由于从网页下载较慢,就放在本仓库了,感谢作者开源贡献。}
    \item \LaTeX{} 教程:
        \begin{itemize}
            \item \href{h ttps://github.com/luong-komorebi/Begin-Latex-in-minutes/blob/master/Translation-Chinese.md}
                {学习 \LaTeX{} 从现在开始 (Begin \LaTeX{} in Minutes)}
            \item
                \href{https://www.ctan.org/tex-archive/info/lshort/chinese}
                {一份不太简短的LATEX介绍}
        \href{http://mirrors.ctan.org/info/lshort/chinese/lshort-zh-cn.pdf}
                {中文下载}
                \footnote{如果 \lstinline{git clone} 了本仓库,
                可以在本仓库的 \lstinline{books}
                目录中阅读此电子书:
                \href{books/lshort-zh-cn.pdf}{一份不太简短的LATEX介绍} 。
                由于从网页下载较慢,就放在本仓库了,感谢作者开源贡献。}
            \item
                \href{https://www.ctan.org/tex-archive/info/lshort/chinese}
                {一份简短的 LaTeX 数学指南}
                \href{https://wenda.latexstudio.net/article-5006.html}
                {中文下载}
                \footnote{如果 \lstinline{git clone} 了本仓库,
                可以在本仓库的 \lstinline{books}
                目录中阅读此电子书:
                \href{books/short-math-guide(cn).pdf}
                {一份简短的 LaTeX 数学指南} 。
                由于从网页下载较慢,就放在本仓库了,感谢作者开源贡献。}
                \href{http://mirrors.ustc.edu.cn/CTAN/info/short-math-guide/short-math-guide.pdf}{英文下载}
                \footnote{如果 \lstinline{git clone} 了本仓库,
                可以在本仓库的 \lstinline{books}
                目录中阅读此电子书:
                \href{books/short-math-guide.pdf}
                {Short Math Guide for \LaTeX{}} 。
                由于从网页下载较慢,就放在本仓库了,感谢作者开源贡献。}
            \item \href{https://github.com/huangxg/lnotes}
                {雷太赫排版系统简介 An introduction to TeX/LaTeX
                typesetting system}
        \href{https://github.com/huangxg/lnotes/raw/master/lnotes2.pdf}{下载}
                \footnote{如果 \lstinline{git clone} 了本仓库,
                可以在本仓库的 \lstinline{books}
                目录中阅读此电子书:
                \href{books/lnotes2.pdf}{雷太赫排版系统简介} 。
                由于从网页下载较慢,就放在本仓库了,感谢作者开源贡献。}
        \end{itemize}
\end{itemize}

%\section{使用教程}%
%\label{sec:latex-tips}

%待续

\chapter{Git 使用教程}%
\label{cha:git-tips}

\section{基本配置}%
\label{sec:git-settings}


\subsection{Github 的注册和 SSH 设置}%
\label{sec:git-ssh}

设置步骤:
\begin{itemize}
    \item 在 \href{https://github.com/}{https://github.com/}
        上登陆(注册并登陆)你的账户。
        假设你在 \lstinline{github.com} 的用户名为 \lstinline{yourUsername},
        你的电子邮件为 \lstinline{yourEmail}.
    \item 新建在github上使用的公匙:
        \begin{itemize}
            \item 打开一个终端 (\lstinline{terminal});
            \item 运行以下命令 (用你注册时的电子邮件账户替换
                \lstinline{yourEmail});
\begin{lstlisting}
    ssh-keygen -t rsa -C "yourEmail" -f ~/.ssh/github_ddpom_ganx
\end{lstlisting}
            \item 提示输入 \lstinline{passphrase}
                时,可以留空,也可以输入你的 \lstinline{passphrase};
            \item 运行以下命令,将公钥复制到剪切板;
\begin{lstlisting}
    xclip -sel clip < ~/.ssh/github_ddpom_ganx.pub
\end{lstlisting}
            \item 粘贴剪切板里的内容到你的 GitHub 账户中的 SSH 公钥:
                \begin{itemize}
                    \item 访问 \href{https://github.com/settings/keys}
                        {https://github.com/settings/keys},
                        如果提示你登陆的话,请登陆;
                    \item 点击右上角的 \lstinline{New SSH key} 后,
                        粘贴剪切板里的内容到文本框。
                \end{itemize}
            \item 运行以下命令,测试你的 SSH 设置:
\begin{lstlisting}
    ssh -T git@github.com
\end{lstlisting}
            \item 如果你看到如下信息,你的设置就成功了。
\begin{lstlisting}
    Hi yourUsername! You've successfully authenticated, but GitHub does not provide shell access.
\end{lstlisting}
        \end{itemize}
\end{itemize}

\newpage
\subsection{在虚拟机中设置用户名和电子邮件}%
\label{sub:git-user-info}

\begin{enumerate}
    \item 用 \lstinline{gvim} 修改 \lstinline{~/.gitconfig_local}
        (如果没有这个文件, \lstinline{gvim} 会新建这个文件并打开它):
\begin{lstlisting}[escapeinside=``]
    gvim ~/.gitconfig_local
\end{lstlisting}
    \item 写入下面的内容,用你个人的用户名和邮件地址替换相应的信息:
\begin{lstlisting}[escapeinside=``]
[user]
	name = yourUsername
	email = yourEmail

[github]
	user = yourUsername
\end{lstlisting}
\end{enumerate}

%\section{Git 的简单介绍}%
%\label{sec:git-intro}
%
%待续
%
%\section{Git 的学习方式}%
%\label{sec:git-learning}
%
%待续
%
%\section{Git 使用教程及技巧}%
%\label{sec:git-tips}
%
%待续

%\nocite{*}

%\bibliography{reference}

%\appendix

\end{document}
