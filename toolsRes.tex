%!TEX TS-program = xelatex
%!TEX encoding = UTF-8 Unicode

\documentclass[cn,11pt, simple]{elegantbook}

\title{Virtualbox、Linux、Vim、\LaTeX{}、Git、Zotero}
\subtitle{科学研究中常用的开源工具的简单介绍}

\author{甘湘华}
\institute{西南财经大学}
\date{\today}
\version{0.01}

\extrainfo{本教程基于已经配置好所用软件的
    {\rm Virtualbox} 虚拟机,力求尽快让技术小白能够用
    {\rm \LaTeX{}} 写研究论文,以 及用~{\rm \lstinline{Git}}
    工具来进行版本控制。
    需要声明的是,作者本人也是小白一枚,请读者批判地学习本教程。
    不当以及需改进之处,请不吝指出,谢谢!}

\logo{logo.png}
\cover{cover.jpg}

\begin{document}

\maketitle

\tableofcontents
% \thispagestyle{empty}

\mainmatter
\hypersetup{pageanchor=true}

\chapter{系统设置}
\label{cha:settings-system}

\section{Virtualbox 虚拟机}%
\label{sec:virtualbox}


\subsection{参考资料}%
\label{sub:virtualbox-refs}

Virtualbox 虚拟机软件参考资料网址:
\begin{itemize}
    \item 官方网址: \href{https://www.virtualbox.org/}{https://www.virtualbox.org/}
    \item 官方论坛: \href{https://forums.virtualbox.org/}{https://forums.virtualbox.org/}
    \item 网络上的非官方网址和非官方教程:
        \begin{itemize}
            \item 如果官方网址下载虚拟机软件慢的话,
                可以在
                \href{https://mirror.tuna.tsinghua.edu.cn/virtualbox/6.0.14/}
                {清华镜像}下载。
            \item \href{https://www.jianshu.com/p/bfb4f4415411}
                {安装 virtualbox 教程}
                \footnote{本教程使用的虚拟机版本是 6.0.14}
                % \footnote{本教程作者已经感谢该文作者。}
        \end{itemize}
\end{itemize}

\subsection{常用操作}%
\label{sub:virtualbox-tips}

\subsubsection{设定共享目录}%
\label{ssub:virtualbox-tip-share-folder}
\begin{figure}[!htbp]
  \centering
  \includegraphics[width=0.9\textwidth]{screenshots/Screenshot from
  2019-11-29 17-52-45.png}
  \caption{设定共享目录 --- 示意图 1}
\end{figure}
\begin{figure}[!htbp]
  \centering
  \includegraphics[width=0.9\textwidth]{screenshots/Screenshot from
  2019-11-29 17-53-28.png}
  \caption{设定共享目录 --- 示意图 2}
\end{figure}
\begin{figure}[!htbp]
  \centering
  \includegraphics[width=0.9\textwidth]{screenshots/Screenshot from
  2019-11-29 17-54-41.png}
  \caption{设定共享目录 --- 示意图 3}
\end{figure}
\begin{figure}[!htbp]
  \centering
  \includegraphics[width=0.9\textwidth]{screenshots/Screenshot from
  2019-11-29 17-55-59.png}
  \caption{设定共享目录 --- 示意图 4}
\end{figure}
\begin{figure}[!htbp]
  \centering
  \includegraphics[width=0.9\textwidth]{screenshots/Screenshot from
  2019-11-29 17-59-22.png}
  \caption{设定共享目录 --- 示意图 5}
\end{figure}

\subsubsection{调整虚拟机窗口大小}%
\label{ssub:virtualbox-tip-window-size}

至少有三种方法:
\begin{enumerate}
    \item 最大化虚拟机窗口。(在 Linux 和 Windows 上通常为点击右上角的方框,
        在 Mac 上点击左上的放大键)
    \item 按组合键 Host + H.(组合键在 Linux 和 Windows 上通常为右边的
        Ctrl, 在 Mac 上通常为左边的 Command 键)
    \item 用鼠标拖动窗口,调节窗口大小。
\end{enumerate}

\section{Linux 系统}%
\label{sec:linux-tips}

\subsection{参考资料}%
\label{sub:linux-refs}

\begin{itemize}
    \item 官方网址: \href{https://www.linux.org/}{https://www.linux.org/}
    \item Ubuntu ~ ~: \href{https://cn.ubuntu.com/}{https://cn.ubuntu.com/}
    \item 社区讨论:
        \href{https://forum.ubuntu.com.cn/}{https://forum.ubuntu.com.cn/}
    \item 问答网站:
        \href{https://askubuntu.com/}{https://askubuntu.com/}
    \item Linux 教程:
        \begin{itemize}
            \item \lstinline{Ubuntu desktop help}
                \footnote{按 \lstinline{F1} 即可以打开。}
            \item \href{https://help.ubuntu.com/lts/ubuntu-help/index.html}
                {Ubuntu 桌面指南}
            \item \href{https://linuxtools-rst.readthedocs.io/zh_CN/latest/}
                {Linux 工具快速教程}
            \item \href{https://dunwu.github.io/linux-tutorial/#/}
                {Linux 教程}
        \end{itemize}
\end{itemize}

\subsection{Linux 系统设置}%
\label{sub:linux-intro}


\subsubsection{点文件~(dotfiles)配置}%
\label{ssub:dotfiles-settings}

\paragraph{点文件介绍}%
\label{sssub:dotfiles-settings-intro}

在 Unix/Linux/Mac 系统上,软件有它们各自的配置文件,通常以 ~.  开头,
管它们叫 \emph{点文件}~(~ \texttt{dotfiles}).
这些点文件数量多,并且所在的位置还有所不同,管理起来并不是一个容易的事情。
一个方法是把所有的这些配置文件放进一个文件夹,对每个配置文件用命令
\lstinline{ln -s} 链接到原来的位置。
这个文件夹可以用 git 仓库进行管理,
可以方便我们把一台电脑上的配置,同步到另外的电脑上。

作者配置的虚拟机上使用
\href{https://github.com/anishathalye/dotbot}{dotbot} 技术来管理 dotfiles.
在本虚拟机的 \lstinline{~/.df/} 目录中有两个文件夹: \lstinline{dotfiles}
中的配置文件是所有电脑中都共同拥有的; \lstinline{dotfiles-local}
中的配置文件是某个电脑所特有的。
建议本教程的读者在未掌握 \lstinline{dotfiles}
的配置前,不要更改这两个文件夹中的任何文件。
\begin{note}\label{note:vim-customized}
    如果要对 Vim 做一些个性化的设置,可以参考 \ref{ssub:vim-config}.
\end{note}

%\subsubsection{使用本虚拟机的注意事项}%
%\label{ssub:virtualbox-guest-ganx}

\paragraph{同步点文件的 Git 仓库}%
\label{sssub:virtualbox-guest-ganx-conf}

本虚拟机的配置文件主要由以下两个仓库控制:
\begin{itemize}
    \item \href{https://github.com/ddpom/dotfiles}
        {https://github.com/ddpom/dotfiles}
    \item \href{https://github.com/ddpom/dotfiles-local}
        {https://github.com/ddpom/dotfiles-local}
\end{itemize}

\paragraph{设置点文件仓库的更新提示}%
\label{sssub:virtualbox-guest-ganx-conf-update-notification}

两个仓库不时更新的,更新配置文件的提交实现后,会发出更新提示。
如果你 \lstinline{watch (关注)} 了这两个仓库,就可以收到更新提示。
设置步骤如下:
\begin{enumerate}
    \item 访问 \href{https://github.com/ddpom/dotfiles}
        {https://github.com/ddpom/dotfiles}, 点击右上方
        \lstinline{watch (关注)} 旁边的下拉菜单,选择 \lstinline{watching}.
        如果你觉得这个仓库还有用的话,顺便点一下右边的 \lstinline{star}.
    \item 访问 \href{https://github.com/ddpom/dotfiles-local}
        {https://github.com/ddpom/dotfiles-local},
        点击右上方 \lstinline{watch (关注)} 旁边的下拉菜单,
        选择 \lstinline{watching}.
        如果你觉得这个仓库还有用的话,顺便点一下右边的 \lstinline{star}.
    \item 访问 \href{https://github.com/settings/notifications}
        {https://github.com/settings/notifications},
        配置你接收提示的选项。缺省设置下,你会在电子邮件里收到你关注的仓库的
        提交。你还可以打钩 \lstinline{Web and Mobile} 选项。
\end{enumerate}

\paragraph{更新点文件 --- 手动方式}%
\label{sssub:update-dotfiles-manual}

\begin{enumerate}
    \item 运行下面几个命令,更新 \href{https://github.com/ddpom/dotfiles}
        {https://github.com/ddpom/dotfiles}:
\begin{lstlisting}[escapeinside=``]
cd ~/.df/dotfiles
git checkout .
git pull --rebase
git submodule update --remote
./install
\end{lstlisting}
    \item 运行下面几个命令,更新
        \href{https://github.com/ddpom/dotfiles-local}
        {https://github.com/ddpom/dotfiles-local}:
\begin{lstlisting}[escapeinside=``]
cd ~/.df/dotfiles-local
git checkout .
git pull --rebase
git submodule update --remote
./install
\end{lstlisting}
\end{enumerate}

\paragraph{更新点文件 --- 自动方式}%
\label{sssub:update-dotfiles-crontab}

\begin{enumerate}
    \item 将 ~\ref{sssub:update-dotfiles-manual}~ 中的所有命令写入
        \lstinline{~/.df/dotfiles-local/scripts/pullDotfiles.sh}.
        \begin{note}\label{note:update-dotfiles-crontab}
            这一步是更新 \lstinline{dotfiles-local} 时
            自动完成,虚拟机使用者不需要手动写入。
        \end{note}
    \item 运行命令 \lstinline{crontab -e}, 写入以下命令:
\begin{lstlisting}[escapeinside=``]
SHELL=/usr/bin/zsh
PATH=/sbin:/bin:/usr/sbin:/usr/bin
MAILTO=""

@hourly  /usr/bin/zsh ~/.df/dotfiles-local/scripts/pullDotfiles.sh
\end{lstlisting}
\begin{note}\label{note:crontab}
    我们可以把自动更新的运行记录写入 \lstinline{log} 文件。
    首先,运行~ \lstinline{mkdir ~/.df/log} 创建 \lstinline{log} 目录;
    然后运行命令 \lstinline{crontab -e}, 在最后一行添加
    \lstinline{ >> ~/.df/log/pullDotfiles.log}:
\begin{lstlisting}[escapeinside=``]
@hourly  /usr/bin/zsh ~/.df/dotfiles-local/scripts/pullDotfiles.sh >> ~/.df/log/pullDotfiles.log
\end{lstlisting}
添加以后不要断行,要确保所有的内容在一行上。
\end{note}
\end{enumerate}

\subsection{本教程所用虚拟机的安装和设置}%
\label{sub:virtualbox-guest-settings}

\subsubsection{虚拟机的基本安装}%
\label{ssub:linux-ubuntu}

本虚拟机的系统为 Ubuntu 18.04; 安装选项为最小安装;用户名: ganx; 密码: 9;
登录方式:自动登录。

\subsubsection{安装虚拟机的 Guest Additions}%
\label{ssub:virtualbox-guest-additions}
\begin{enumerate}
    \item \lstinline{sudo apt update}
    \item \lstinline{sudo apt install build-essential gcc make perl dkms menu}
    \item 打开安装盘,点击用于 Linux 系统的可执行文件直接安装。
\end{enumerate}

\subsubsection{在主机上设置与虚拟机的目录共享}%
\label{ssub:vitualbox-shared-folder}
\begin{enumerate}
    \item \lstinline{sudo usermod -a -G vboxsf ganx}
    \item 参考 \ref{ssub:virtualbox-tip-share-folder},
        设置guest上和主机目录对应的共享目录 (注意: 勾选自动挂载)。
    \item 重启虚拟主机或者运行命令 \lstinline{sudo systemctl daemon-reload},
        共享目录生效。
\end{enumerate}

\subsubsection{Virtualbox 的其它设置}%
\label{ssub:virtualbox-others}
\begin{enumerate}
    \item 调整虚拟机使用的内存大小;
    \item 设置虚拟机的共享剪贴板选项为 \emph{双向};
    \item 设置用户界面。
\end{enumerate}


\subsubsection{安装 i-bus 拼音}%
\label{ssub:pinyin-i-bus}
\begin{enumerate}
    \item \lstinline{sudo apt install ibus-libpinyin pinyin-database}
    \item 安装中文语言。
    \item 将 "intelligent pinyin" 加入 "Input source".
\end{enumerate}

\subsubsection{安装搜狗拼音}%
\label{ssub:pinyin-sogou}
\begin{enumerate}
    \item 在 \href{https://pinyin.sogou.com/linux/}
        {https://pinyin.sogou.com/linux/} 下载 64 位版本的安装包。
        \footnote{安装本虚拟机时下载的版本是
            \lstinline{sogoupinyin_2.3.1.0112_amd64.deb}
        }
    \item \lstinline{sudo apt -y install fcitx fcitx-bin fcitx-table-all}
    \item \lstinline{sudo dpkg -i sogoupinyin_2.3.1.0112_amd64.deb}
    \item \lstinline{sudo apt-get install -f}
    \item 在系统设置中的语言设置选择缺省输入方式为 fcitx.
\end{enumerate}

\subsection{常用命令}%
\label{sub:linux-commands}

常用命令:
\begin{enumerate}
    \item \lstinline{cd}: 进 入某个目录; 例子:
\begin{lstlisting}[ escapeinside=``]
cd ~/.ssh/
\end{lstlisting}
    \item \lstinline{mv}: 1. 修改文件或这目录的名字; 2.
        移动文件或者整个目录; 例子:
\begin{lstlisting}[escapeinside=``]
cd ~/.ssh/
mv github_yourname_ganx github_ddpom_ganx
mv github_yourname_ganx.pub github_ddpom_ganx.pub
\end{lstlisting}
\end{enumerate}

\section{Vim 编辑器}%
\label{sec:vim}

\subsection{参考资料}%
\label{sub:vim-refs}

\begin{itemize}
    \item 官方网址: \href{https://www.vim.org/}{https://www.vim.org/}
    \item 问答网站:
        \href{https://vi.stackexchange.com/}{https://vi.stackexchange.com/}
    \item Vim 教程:
        \begin{itemize}
            \item \lstinline{vimtutor}
                \footnote{在终端中,输入 \lstinline{vimtutor} 命令学习 Vim
                的基本用法。}
            \item \href{https://github.com/wsdjeg/vim-galore-zh_cn}
                {Vim 从入门到精通}
            \item \href{https://item.jd.com/12056490.html}
                {Vim 使用技巧 (Pratical Vim) 第二版}
        \end{itemize}
\end{itemize}

\subsection{常用配置}%
\label{sub:vim-intro}

\subsubsection{Vim 的配置文件}%
\label{ssub:vim-config}

本虚拟机中的配置文件为:
\begin{lstlisting}[escapeinside=``]
~/.df/dotfiles/vimrc
~/.df/dotfiles-local/gvimrc
~/.vimrc_customized
~/.gvimrc_customized
\end{lstlisting}
\lstinline{vimrc} 和 \lstinline{gvimrc} 由
\ref{sssub:virtualbox-guest-ganx-conf} 中的两个 git 仓库控制,请勿修改。
你对本台虚拟机的特殊设置可以在 \lstinline{.vimrc_customized} 和
\lstinline{.gvimrc_customized} 中配置。

\subsubsection{Gvim 的个性化设置}%
\label{ssub:vim-gui-config-customized}

打开 \lstinline{~/.gvimrc_customized}, 进行 Gvim 的个性化设置。例如:
\begin{lstlisting}[escapeinside=``]
set guifont=Monospace\ 13
set linespace=4
\end{lstlisting}
对于以上两个选项的设置,可以用命令 \lstinline{:h guifont} 和
\lstinline{:h linespace} 查看帮助。
例如,在 \lstinline{guifont} 的设置帮助中,有建议在中文系统下,使用如下配置:
\begin{lstlisting}[escapeinside=``]
if has("gui_gtk3")
  set guifont=Bitstream\ Vera\ Sans\ Mono\ 12,Fixed\ 12
  set guifontwide=Microsoft\ Yahei\ 12,WenQuanYi\ Zen\ Hei\ 12
endif
\end{lstlisting}
\begin{note}\label{note:guifont}
    在本虚拟机上,这个建议的设置不尽人意。也许在 \lstinline{Window}
    系统上还不错。
\end{note}


\subsection{故障排除}%
\label{sub:vim-troubleshooting}

\subsubsection{Gvim 打开文件后不见光标}%
\label{ssub:vim-ts-cursor}

\begin{enumerate}
    \item 最大化虚拟机窗口(参见
        \ref{ssub:virtualbox-tip-window-size})后打开文件,
        若能见光标就停止,否则进入下一步。
    \item 打开 \lstinline{~/.gvimrc_customized}, 修改
        \lstinline{linespace} 的设置。比如,可以尝试  \lstinline{0 - 4}
        之间的一个值。
\end{enumerate}


%待续
%%Vim 的学习曲线(To do)。
%
%\subsection{Vim 的学习方式}%
%\label{sub:vim-learning}

\chapter{\LaTeX{} 使用教程}%
\label{cha:latex-tips}

\section{参考资料}%
\label{sec:latex-refs}

\LaTeX{} 参考资料网址:
\begin{itemize}
    \item 官方网址: \href{https://www.ctan.org/}{https://www.ctan.org/}
    \item 问答网站:
        \href{https://tex.stackexchange.com/}{https://tex.stackexchange.com/}
    \item 软件下载:如果官方网址下载 \lstinline{texlive.iso} 慢的话,可以在
    \href{https://mirrors.tuna.tsinghua.edu.cn/CTAN/systems/texlive/Images/}
                {清华镜像}~下载。
    \item 数学符号: \href{books/latex-math-symbols.pdf}{速查表}~ ~
        \href{http://detexify.kirelabs.org/classify.html} {图像识别} ~ ~
        \href{http://mirrors .ustc.edu.cn/CTAN/info/symbols/comprehensive/symbols-a4.pdf}
        {详细列表}
        \footnote{如果 \lstinline{git clone} 了本仓库,
        可以在本仓库的 \lstinline{books}
        目录中阅读此电子书:
        \href{books/symbols-a4.pdf}{数学符号详细列表} 。
        由于从网页下载较慢,就放在本仓库了,感谢作者开源贡献。}
    \item \LaTeX{} 教程:
        \begin{itemize}
            \item \href{h ttps://github.com/luong-komorebi/Begin-Latex-in-minutes/blob/master/Translation-Chinese.md}
                {学习 \LaTeX{} 从现在开始 (Begin \LaTeX{} in Minutes)}
            \item
                \href{https://www.ctan.org/tex-archive/info/lshort/chinese}
                {一份不太简短的LATEX介绍}
        \href{http://mirrors.ctan.org/info/lshort/chinese/lshort-zh-cn.pdf}
                {中文下载}
                \footnote{如果 \lstinline{git clone} 了本仓库,
                可以在本仓库的 \lstinline{books}
                目录中阅读此电子书:
                \href{books/lshort-zh-cn.pdf}{一份不太简短的LATEX介绍} 。
                由于从网页下载较慢,就放在本仓库了,感谢作者开源贡献。}
            \item
                \href{https://www.ctan.org/tex-archive/info/lshort/chinese}
                {一份简短的 LaTeX 数学指南}
                \href{https://wenda.latexstudio.net/article-5006.html}
                {中文下载}
                \footnote{如果 \lstinline{git clone} 了本仓库,
                可以在本仓库的 \lstinline{books}
                目录中阅读此电子书:
                \href{books/short-math-guide(cn).pdf}
                {一份简短的 LaTeX 数学指南} 。
                由于从网页下载较慢,就放在本仓库了,感谢作者开源贡献。}
                \href{http://mirrors.ustc.edu.cn/CTAN/info/short-math-guide/short-math-guide.pdf}{英文下载}
                \footnote{如果 \lstinline{git clone} 了本仓库,
                可以在本仓库的 \lstinline{books}
                目录中阅读此电子书:
                \href{books/short-math-guide.pdf}
                {Short Math Guide for \LaTeX{}} 。
                由于从网页下载较慢,就放在本仓库了,感谢作者开源贡献。}
            \item \href{https://github.com/huangxg/lnotes}
                {雷太赫排版系统简介 An introduction to TeX/LaTeX
                typesetting system}
        \href{https://github.com/huangxg/lnotes/raw/master/lnotes2.pdf}{下载}
                \footnote{如果 \lstinline{git clone} 了本仓库,
                可以在本仓库的 \lstinline{books}
                目录中阅读此电子书:
                \href{books/lnotes2.pdf}{雷太赫排版系统简介} 。
                由于从网页下载较慢,就放在本仓库了,感谢作者开源贡献。}
        \end{itemize}
\end{itemize}

%\section{使用教程}%
%\label{sec:latex-tips}

%待续

\chapter{Git 使用教程}%
\label{cha:git-tips}

\section{参考资料}%
\label{sec:git-refs}

\begin{itemize}
    \item 官方网址: \href{https://git-scm.com/}{https://git-scm.com/}
    \item 问答网站:
        \href{https://stackoverflow.com/questions/tagged/git+github}
        {https://stackoverflow.com/questions/tagged/git+github}
    \item Git~ 教程:
        \begin{itemize}
            \item \href{https://git-scm.com/book/zh/v2}
                {Pro Git 中文}
                ~ \href{books/progit_v2.1.31_zh.pdf}{pdf 中文}
                ~ \href{https://git-scm.com/book/en/v2} {Pro Git 英文}
                 ~ \href{books/progit_v2.1.177_en.pdf}{pdf 英文}
            \item \href{https://help.github.com/en/github}
                {Github 帮助}
        \end{itemize}
\end{itemize}
\section{常用配置}%
\label{sec:git-settings}

\subsection{Github 的注册和 SSH 设置}%
\label{sub:git-ssh}

设置步骤:
\begin{itemize}
    \item 在 \href{https://github.com/}{https://github.com/}
        上登陆(注册并登陆)你的账户。
        假设你在 \lstinline{github.com} 的用户名为 \lstinline{yourUsername},
        你的电子邮件为 \lstinline{yourEmail}.
    \item 新建在github上使用的公匙:
        \begin{itemize}
            \item 打开一个终端 (\lstinline{terminal});
            \item 运行以下命令 (用你注册时的电子邮件账户替换
                \lstinline{yourEmail});
\begin{lstlisting}
ssh-keygen -t rsa -C "yourEmail" -f ~/.ssh/github_ddpom_ganx
\end{lstlisting}
            \item 提示输入 \lstinline{passphrase}
                时,可以留空,也可以输入你的 \lstinline{passphrase};
            \item 运行以下命令,将公钥复制到剪切板;
\begin{lstlisting}
xclip -sel clip < ~/.ssh/github_ddpom_ganx.pub
\end{lstlisting}
            \item 粘贴剪切板里的内容到你的 GitHub 账户中的 SSH 公钥:
                \begin{itemize}
                    \item 访问 \href{https://github.com/settings/keys}
                        {https://github.com/settings/keys},
                        如果提示你登陆的话,请登陆;
                    \item 点击右上角的 \lstinline{New SSH key} 后,
                        粘贴剪切板里的内容到文本框。
                \end{itemize}
            \item 运行以下命令,测试你的 SSH 设置:
\begin{lstlisting}
ssh -T git@github.com
\end{lstlisting}
            \item 如果你看到如下信息,你的设置就成功了。
\begin{lstlisting}
Hi yourUsername! You've successfully authenticated, but GitHub does not provide shell access.
\end{lstlisting}
        \end{itemize}
\end{itemize}

\subsection{在虚拟机中设置用户名和电子邮件}%
\label{sub:git-user-info}

\begin{enumerate}
    \item 用 \lstinline{gvim} 修改 \lstinline{~/.gitconfig_local}
        (如果没有这个文件, \lstinline{gvim} 会新建这个文件并打开它):
\begin{lstlisting}[escapeinside=``]
gvim ~/.gitconfig_local
\end{lstlisting}
    \item 写入下面的内容,用你个人的用户名和邮件地址替换相应的信息:
\begin{lstlisting}[escapeinside=``]
[user]
	name = yourUsername
	email = yourEmail

[github]
	user = yourUsername
\end{lstlisting}
\end{enumerate}

%\section{Git 的简单介绍}%
%\label{sec:git-intro}
%
%待续
%
%\section{Git 的学习方式}%
%\label{sec:git-learning}
%
%待续
%
\section{常用操作}%
\label{sec:git-tips}
%
%待续

\chapter{论文写作工具}%
\label{cha:tools-writing}

\section{Zotero 文献管理}%
\label{sec:citation-zotero}

\section{Ludwig 英文语句搜索引擎}%
\label{sec:writing-ludwig}

Ludwig 英文语句搜索引擎可以帮助研究者选择适用的词句。现在以一个标题为例:
Solving a Perishables Joint Pricing and Inventory Control Problem
with Positive Lead Time via Deep Reinforcement Learning.
对于一个以英语为母语或者工作语言的人来看,这个标题显然是有点问题的。
对于不常用英文写作的研究者,可以考虑使用
\href{https://ludwig.guru/}{ludwig.guru/} 来帮助英文写作.

如果我们对 Perishables Joint Pricing and Inventory Control
不确定正确与否,可以在搜索框中查看英文文献中对这个短语的使用情况,参见图
\ref{fig:ludwig-search}。

\begin{figure}[!htbp]
  \centering
  \includegraphics[width=1.0\textwidth]{screenshots/Screenshot from
  2019-12-23 21-13-03.png}
  \caption{搜索 Perishables Joint Pricing and Inventory Control}
  \label{fig:ludwig-search}
\end{figure}

从图 \ref{fig:ludwig-result-1} 及 \ref{fig:ludwig-result-2}
可以看到,搜索的结果不理想。
因为 Ludwig 使用的语言资料库太大,搜索结果不理想时可以考虑缩小搜索范围。
注意到缺省情况下筛选功能是关闭了的(可以看到 filter OFF
字样),我们可以打开筛选选项,根据实际情况选择搜索范围。详情参见图
\ref{fig:ludwig-filter}.

令人遗憾的是,搜索结果任然不理想(见图
\ref{fig:ludwig-result-3}),没有看到完整的匹配,只看到左中位置显示的 60
SIMILAR.
这时,我们可以考虑搜索更少的关键词。
从搜索结果中,我们发现 Perishables
和其它关键词甚少关联,可以考虑去掉这个关键词后重新搜索。
搜索结果见图 \ref{fig:ludwig-result-4},
我们可以看到仍然没有得到完美匹配,只看到 60 SIMILAR.
但令人欣慰的是,在第一个结果中,我们看到了除大小写外的完美匹配。
可以理解成这个搜索是大小写敏感的。
这时我们可以考虑把搜索框中的大写字母全部改成小写字母来搜索。
搜索结果见图 \ref{fig:ludwig-result-5}, 我们可以看到两个完美匹配
(2 EXACT).

\begin{figure}[!htbp]
  \centering
  \includegraphics[width=1.0\textwidth]{screenshots/Screenshot from
  2019-12-23 21-16-43.png}
  \caption{搜索结果 1}
  \label{fig:ludwig-result-1}
\end{figure}

\begin{figure}[!htbp]
  \centering
  \includegraphics[width=1.0\textwidth]{screenshots/Screenshot from
  2019-12-23 21-16-55.png}
  \caption{搜索结果 2}
  \label{fig:ludwig-result-2}
\end{figure}

\begin{figure}[!htbp]
  \centering
  \includegraphics[width=1.0\textwidth]{screenshots/Screenshot from
  2019-12-23 21-17-32.png}
  \caption{筛选搜索范围}
  \label{fig:ludwig-filter}
\end{figure}

\begin{figure}[!htbp]
  \centering
  \includegraphics[width=1.0\textwidth]{screenshots/Screenshot from
  2019-12-23 21-18-09.png}
  \caption{搜索结果 3}
  \label{fig:ludwig-result-3}
\end{figure}

\begin{figure}[!htbp]
  \centering
  \includegraphics[width=1.0\textwidth]{screenshots/Screenshot from
  2019-12-23 21-19-22.png}
  \caption{去掉 Perashables 后的搜索结果}
  \label{fig:ludwig-result-4}
\end{figure}

\begin{figure}[!htbp]
  \centering
  \includegraphics[width=1.0\textwidth]{screenshots/Screenshot from
  2019-12-23 21-20-01.png}
  \caption{去掉 Perashables 后且使用小写字母的搜索结果}
  \label{fig:ludwig-result-5}
\end{figure}

%\nocite{*}

%\bibliography{reference}

%\appendix

\end{document}
